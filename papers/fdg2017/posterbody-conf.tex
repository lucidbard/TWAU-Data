\section{Story Control Logics}
\label{sec:orgheadline6}
Noah Wardrip-Fruin introduced the analytical concept of operational
logics \cite{Wardrip-fruin} to describe the synergistic relationship
of abstract process and representational goals. He did so in the
context of understanding the relationships between surface, process
and data in the work leading up to the book \emph{Expressive Processing},
and described some of the key insights there \cite{Wardrip-Fruin2009}.
The concept was later formalized and further defined in collaboration
with Michael Mateas in \cite{Mateas2009b}:

\begin{quote}
An operational logic defines an authoring (representational)
strategy, supported by abstract processes or lower-level logics, for
specifying the behaviors a system must exhibit in order to be
understood as representing a specified domain to a specified
audience. \cite{Mateas2009b}
\end{quote}

Operational logics can be used to precisely model and analyze how an
author communicates through a set of abstract processes and
representations an underlying "domain" with an audience. This
proposal's surface content model depends on the slipperiness of the
domain of human affairs that is narrative, and how at the same time
operational logics "determine the state evolution of a system," how
they specify an abstract model of the underlying system, "with how
they are understood at a human level," through the proposed encoding
\cite{Mateas2009b}. Some examples of operational logics that are
proposed by Wardrip-Fruin include collision logics and dialogue tree
logics, which reflect certain strategies to manage the content and the
actual experience of engaging with it \cite{Wardrip-Fruin2009}. As
long as a particular logic is maintained throughout a work, and that
logic reflects a systematic application of an underlying process, it
can be fruitfully analyzed as an operational logic. In this section,
the operational logics targeted are principally those that control and
represent the branching of the story and not necessarily logics that
pertain to any underlying simulation of the story domain. In this way,
they are not that much different than dialogue tree logics or the use
of quest flags, and the proposed SIG extension takes advantage of that
simplicity. The main difference is that the strategy of reflecting the
choices through interfaces are tied to an underlying time-based
performance that is critical to the pacing of the dramatic work.

The encoding of the surface representation depends on a model of the
underlying operations that each of the proposed story control
logics. These logics occupy the intersection between (graphical)
interface logics and performance logics, as they provide the player
with options tied to an underlying content selection architecture as
well as afford the player a set of performance options that either
determine or enact the choices.

The state evolution and representational strategy of cinematic
choice-based adventure games is represented by modeling the following
\textbf{story control logics}:

\begin{center}
\begin{tabular}{ll}
Operational Logic Name & Description\\
\hline
\textbf{1} \emph{Response Selection} & Language-based menus that reflect player-character options.\\
\textbf{2} \emph{Object-Verb Selections} & Verb-selection for object or character.\\
\textbf{3} \emph{Inventory} & Objects that either reflect ongoing or past plots\\
\textbf{4} \emph{Quick-time} & Affordances that require enactment to proceed\\
\textbf{5} /"Payoff" Signaling/ & Text that indicates (truthfully or not) underlying state\\
 & \\
\end{tabular}
\end{center}

There are other operational logics at work, such as collision logics
and camera control logics, but these don't affect the story structure
as defined by the SIG. The four above operational logics cover every
relevant non-linear control mechanism and corresponding
representation. These logics must be reflected in a surface model in
order to represent the possible variations. Because of the importance
of these control logics in determining the surface structure, we will
review each in the following sections.

\subsection{Response Selection}
\label{sec:orgheadline1}
Response selection is so named because it how a player selects the
next actions or intent of the character. Like most adventure games,
the character itself actually carries out the action or dialogue, and
even sometimes the way a particular choice plays out may differ from
what was anticipated. The underlying operation is simply one of
selecting the content to be played.

\subsection{Object-Verb Selection}
\label{sec:orgheadline2}
This logic is a variant on the response selection, except that it
plays the character interacting with an object instead of interacting
with a character and often has physical actions instead of dialogue
options. An example is either knocking or kicking down a door, or
picking up an object.

\subsection{Inventory Logics}
\label{sec:orgheadline3}
In this logic, the main function is as a flag to whether certain
content choices are available. Secondarily, it is a reminder of
important plot threads.
\subsection{Quick-time}
\label{sec:orgheadline4}
This represents the enactment of the player of a particular situation
the character is in, often through repeated button presses or through
a quick decision in a heated moment. Failing these may result in
replaying from a checkpoint, so most do not branch the story.
\subsection{/"Payoff"/ signaling}
\label{sec:orgheadline5}

This is one of the more complex logics that may not determine
underlying state at all, but rather simply shaping player expectations
of how their decision affected agents. It is because of this
expectation that it needs to be represented as a distinct logic in the
surface story content, and because it is consistently applied in a way
that would suggest an underlying model is being affected.

The complexity of navigating the surface content defined by these
logics and the recorded gameplay motivates the creation of a "story
browser" tool to aid analyzing the distribution and patterns that the
content takes. This tool is discussed further in section
\url{../../../localhost/static/advancement/proposal/murrayAdvancement.org}.
