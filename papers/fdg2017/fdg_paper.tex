\section{Introduction}
\label{sec:orgheadline2}
Interactive storytelling uses the capabilties of computational media
to dynamically assemble stories based on player input and/or on an
underlying simulation of a world. These narratives can be in any of a
variety of immersion levels: text-based were popularized as the genre
of interactive fiction, whereas the interactive drama \emph{Facade}
\cite{Mateas2005} uses conventions from theater and simulates a
mimetic illusion of a world where characters act and react as agents
with audible natural language in continuous time and space. Regardless
of immersion level, narrative is an important vehicle for
communicating experiences. The present study uses computational models
of narrative to encode a particular storygame, \emph{The Wolf Among Us} by
Telltale games. It was selected due to its nuanced narrative and
critical acclaim, as well as the thriving fanbase. We hypothesized
that the work's intertwining character goals and complicated
relationships with each other and the player would provide an ideal
testing ground for assessing the value of the techniques on
understanding the specific ways in which a specific work negotiated
interactivity and choice within a fictional universe and with a
defined cast. The goal of this paper is to begin the work of applying
formal models to nontrivial works and to embrace the ambiguity,
complexity and nuance in the surface and in the attention to the
narrative simulation taking place.

This work builds on the ongoing efforts within the computational
narratology community, especially those focused on corpuses and formal
models. Modern structural approaches such as those taken by many
narrative generators owe a debt to the original corpus study of
Russian folk tales by Vladimir Propp \cite{Propp1928}.  Noah
Wardrip-Fruin describes computational media in terms of three
components: data, process and surface \cite{Wardrip-Fruin2009a}. Works
such as Telltale Games relies heavily on the narrative and less on
process than simulation intensive works such as the dynamic social
practices simulated in Emily Short and Richard Evan's \emph{Versu}
\cite{Evans2014} or the social games that make up McCoy et al's \emph{Prom
Week} \cite{McCoy2014a}. David Elson intended the \emph{Story Intention
Graph} to be a descriptive model of meaning, representing the mental
simulation that naturally takes place of agents, their pursuit of
goals and resulting causally linked events that make up those
pursuits.

This paper is organized as follows: First, we define the specific
subgenre of cinematic choice-based adventure games and its suitability
for applying computational models of narrative. We then review the
related efforts in computational narratology and game studies. Next,
we describe in detail the selected methodology of preparing a complex
artifact for encoding along with a discussion of the trade-offs
made. We then review the results of the encoding effort with a focus
on lessons learned. Finally, we discuss direction for future work and
conclusions.

\subsection{Commercially succesful choice-based narrative games}
\label{sec:orgheadline1}

Telltale Game's \emph{The Wolf Among Us} was released in 2013 for multiple
platforms. The game received numerous positive critical reviews and
represents a mature example of this subgenre. Its story content is
conveyed through a combination of dialogue, performance of the
characters and cinematography. The game has even been translated into
a comic book, the original media that of the \emph{Fables} series on which
the game is based \cite{Sturges2014}. This document locates the work
in a subgenre of adventure game, and further places it on two axes.

It is useful to locate the specific space that the subgenre of
cinematic choice-based adventure games occupies: where story content
is coded to be presented to the player in a very specific order and
under very specific conditions and where virtually no unexpected
sequence of content occurs.\footnote{This is influenced by Michael Mateas' concept of "content
selection architectures" as a useful way to describe how various types
of content and the configuration of the systems that organize them can
be compared. \cite{Mateas2015}} The label "cinematic choice-based
adventure game" was chosen not because it is the popular name used for
this subgenre of adventure games, which could be considered a hybrid
of the point-and-click adventure game and interactive movie, but
rather for how it highlights the primary mechanics (making choices)
and the primary representational mode (cinematic) as well as including
the historical genre of adventure games which is distinguished by
distinct authored player-character(s) and an emphasis on story.

The adventure game genre is often put in opposition to genres which
promote more player freedom, such as in Massively Multiplayer Role
Playing Games (MMORPGs) or Open World RPGs which allow players to
create and develop their own character. Clara Fernandez-Vara describes
in her dissertation how adventure games shape "the means by which the
player restores the behavior that is expected by playing the game"
\cite{ClaraFernandez-Vara2009}. In other words, in adventure games,
the player takes actions that are authored by the game's creator
rather than have those actions emerge from the system's behaviors. The
player's choices vary the way in which that performance is carried out
as well as certain key facts that make up the plot. Most decisions and
actions only have an impact on the pacing or variation of the
performance rather than ontological changes in the plot, such as the
timing for choosing menu options or choosing two options that have the
same outcome but which have apparent different descriptive text. Some
choices result in content which provides the player satisfaction later
in the game, labeled the "payoff" in this document.

\emph{The Wolf Among Us} is episodic: future episodes must account for
selected previous player decisions, although these are usually limited
to decisions that have an ontological effect on the world (including
the memories of the characters). Episodic storygames have limited
replay value due to the amount of content that is necessarily
conserved. Other non-episodic games, such as \emph{Heavy Rain},\footnote{Quantic Dream, 2010} are
closely related in their operational logics and their suitability for
discovering content relationships using the motheds detailed here.

Games in this subgenre conserve content and maximizing narrative
payoffs among all possible traversals, this subgenre is ideally suited
to annotation using SIG, as the player's goals and intentions are
rewarded for small perturbations while the story remains relatively
consistent.

\section{Related Works}
\label{sec:orgheadline3}
Narrative can be understood as a phenomena that arises from the
coordination of inherent mental abilities, including the ability to
understand the interaction of agents, their goals, and beliefs and the
sequence of causally related events they are involved in. There is a
long history of its study in the field of narratology. 

Interactive Narratives has traversed many traditional silos, offering
both a theoretical challenge to traditional definitions of narratology
and complicating the definitions of game with extra-ludic
meaning. Game critics have developed theory to describe the effects
and goals of study for computational media, including Janet Murray
with \emph{Hamlet on the Holodeck} and Noah Wardrip-Fruin's assessment of
rhetorical strategies employed in procedural works in \emph{Expressive
Processing}\cite{Wardrip-Fruin2009}. Espen Aarseth attempts to
delineate a topology of "ludo-narrative works" and avoids the terms
games \cite{Aarseth2012}. He proposes the dimensions of WORLD,
OBJECTs, AGENTS and EVENTS as ways of plotting and understanding such
works. This locates the subgenre of interest toward the narrative
pole, with the typical adventure game treatment of objects but more
restrictive use of WORLD and EVENTS.

Locating the work in these dimensions does not provide insight into
how to go about critiquing the work's narrative choices in the context
of the use of objects or in the presence of multiple possible
stories. The concept of a world which has topographical significance
is less important than a world which as significance as a
signifier. For example, the extent to which Bigby is able to traverse
the tenement building and his kitchen is far less significant than the
characterization of class they embue or the desires of the
player-character communicated through the affordances of the
environment. A character summoned to deal with a disturbance
eventually must go upstairs to meet it.

In the field of Computational Narratology, Mark Finlaysen conducted a
study of the use of corpuses that observes that text is prioritized
given the availability of tools \cite{Finlayson2013}, although the
only game logs that he cited were those of Orkin in the game EAT \& RUN
\cite{Orkin2010}. 

The literature does not, however, provide an example of a corpus of a
modern interactive narrative game in a format suitable for annotation,
nor does it detail an effort to map an existing model of computational
narratives onto a pre-existing long-form interactive digital
narrative work.

\section{Story Intention Graphs}
\label{sec:orgheadline4}
The lack of contemporary narrative games as sources of study is driven
home in the proliferation of works that both look at and generate
stories of the complexity of Aesop's fables, including the \emph{DramaBank}
developed as part of the assessment of Elson's annotation tool
\emph{Scheherazade} and the Story Intention Graph. Elson sought to validate
the SIG's potential for discovering analogies amongst stories and
Aesop's fables provided a reasonable length and complexity to do
so. Elson includes in the DramaBank a single extended work, \emph{Beowulf},
that was encoded by a single annotator, but the limited representation
of complexity is one of the goals of the present approach. As
Finlaysen laments, "it is easier to use [an existing] corpus than
build another one."  \cite{Finlayson2013}

The Story Intention Graph (SIG) schemata was developed by David Elson
as a set of discourse relations to represent key relationships among
concepts such as goals, values and agents present in textual
narratives using concepts from narrative theory. It consists of three
layers: a \textbf{textual layer}, which contains relevant (but not
exhaustive) spans of text from the source textual story. These are
connected to a set of propositions and states in a layer that captures
the described happenings as a \textbf{timeline layer}. Finally, there is an
\textbf{interpretive layer}, where propositions are linked to agent goals,
plans and values. 

Elson found that the SIG schemata, even without representing
individual propositions, was more successful than alternative methods
at identifying similarities in the stories.

We focus our attention on a subgenre of cinematic choice-based
adventure games. These games are relatively hand-authored and rely on
a series of choices dramatically presented to players. They represent
is an opportunity to understand story and interactivity in a genre
that can be made constrained so as to be tractable for computational
narratology approaches.

This subgenre is an example of a branching narrative, the most common
way of reading and executing non-linear stories. There are a number of
tools now available to create narratives based on a model of lexia and
links, including Ink, Twine, Ren'Py and ChoiceScript. These tools
enable authors to create textual or visual narratives with various
mechanisms to direct the player along particular paths, or traversals.
\section{Methodology}
\label{sec:orgheadline5}
This section describes the steps taken to conform the work into a
format that can be annotated, and to do so in a way that enables
additional work to expand that form without having to redo previous
work. 

\emph{The Wolf Among Us} has not been subject to this type of analysis, as
it is both a commercially available work, it is interactive, and it is
not textual. As a result, some of the terminology and methods
developed for interactive fiction needs to be expanded to account for
the effects that are achieved through the dramatic mode and for the
study of existing interactive narratives in a non-textual medium.

The following requirements for an intermediate format emerged:
\begin{enumerate}
\item It be in a text format, given availability of SIG annotation software
\item Accumulate additional traversals without redoing work, allowing
comparison different traversals and to associate SIG annotations
with the source content rather than the rendered content.
\end{enumerate}

Conducting a rational reconstruction of a game of the size of \emph{The
Wolf Among Us} is challenging. The initial approach is reconstruct the
logic of a single playthrough as if it were a multi-linear
narrative in a scripting language. This has the disadvantage of potentially missing
relationships between content. Many times the exact point at which a
branch merges with the content from other branches is an indication of
a kernel as opposed to a satellite of meaning, using Seymour Chatman's
terminology that positions kernels as essential parts of a story and
satellites as optional\cite{Chatman1980}.

The highest granularity was applied to the first scene, involving
Bigby's first encounter with a Fable, Mr. Toad. In it, the protagonist
and viewpoint character, Bigby Wolf, is called to deal with a
disturbance by a landlord of a tenement building. After arriving,
Bigby discovers the landlord has been disobeying a community rule
about concealing his true identity. 

The density of the initial scene reflects both the interest of the
work as well as its challenge, as the concepts of glamor, class,
shame, and justice are all brought into a simple exchange in the first
few minutes of the game. The scene revolves around both communicating
the nature of the law and the relationship of Mr. Toad to the
community as a person requiring a burdensome cost of a "glamour", or
an enchanted object that can disguise one Fable as another. This
concept is used both to provide plot twists, when a Fable is glamored
to appear as one of the primary characters and appears dead, as well
as used to drive the active theories present in the fan community
about the potential for certain characters to have been glamoured as
others at key moments in the story. 

After the time-consuming modeling of propositions within the first
scene, we decided to heed Elson's results that stated that most of the
success in finding analogies occurred with just interpretive layers
and that propositions resulted in overlapping meaning and low
inter-annotator agreement\cite{Elson2012}. Instead we focused on the
interpretation layer for the following two granularities. The burden
of annotation was described in both Elson's dissertation and
Finlayson's survey and it motivated the creation of both \emph{Scheherazade}
and Finlayson's \emph{Story Workbench} tool for annotating layers onto
textual stories\cite{Finlayson2011}.

The next level of granularity was to capture an entire episode using a
"natural traversal." This was made possible by existing playthroughs
recorded by fans and some by professional streamers. A YouTube video
was selected \cite{wolfamongusstream} based on its completeness as
well as the presence of additional commentary by the player. The
creator-performer was Felix Arvid Ulf Kjellberg, better known as
"PewDiePie". This playthrough video served as the source for a
rational reconstruction of the underlying model in textual format. 

Why is it that a transcript would not do? For a single transcription,
a single transcript of the events that took place could potentially
suffice, including listing choices. However, the next phase will
involve recording additional traversal paths and to further delineate
relationships of content. Without a model of how the content is
rendered, two separate traversals may look too dissimilar to one
another to be able to compare, and the annotation task would scale
with the number of traversals rather than have some effort conserved
with each additional annotation. These and the ability to tabulate
various metadata led to a selection of a format that would both serve
as a plain-text transcription for the annotiation tool as well as
serve as a basis for expanding the transcription to incorporate
additional content from other traversal paths. 

Inkl released an open source version of their narrative scripting
language, \emph{Ink}. \emph{Ink} met all of the criteria, although its exact syntax can vary
significantly and thus offers challenges for certain types of
annotation. The same language was used in the scripting of 80
days\cite{InklLtd.2014}. 

Ink provided the following advantages as a representational format to
meet the second requirement:

\begin{itemize}
\item It is capable of representing simple state variables (whether a
certain choice had been visited/selected and conditionals based on
that). This is used in the transcription.
\begin{itemize}
\item Provide simple means of applying metadata such as whether a line is
dialogue or description, and which character was involved.
\item Simplified notation of converging threads and player choices, used
extensively in the logic of this subgenre.
\item Easy-to-read text format source code for using existing SIG
modeling tool, \emph{Scheherazade}.
\end{itemize}
\end{itemize}

The third level of granularity was applied to the work as a whole, and
represents the overall narrative as modeled using the interpretive
levels of the Story Intention Graph. A similar approach was applied by
a student to Beowulf, taking approximately 15 hours. This effort was
done using a scene-by-scene summary of the same YouTube seriese as the
second level of granularity. The plot summaries from the
fan-maintanied Fables Wiki (\cite{fableswiki}) had sufficient core
story elements for a first encoding and used them as textual sources
in the \emph{Scheherazade} tool. One advantage of using the fan-created plot
summaries was how alternative paths were explicitly detailed, though
this ontological variation introduced its own complications that are
detailed in the next section.


\section{Preliminary Results}
\label{sec:orgheadline13}
This section describes the initial results of applying the methodolgy,
and documents the problems and insights that arose. There are two main
areas where the encoding most benefitted: the delineation of the
various "fan theories" as to the ultimate explanation behind the
events, and a greater appreciation for the means with which the
authors have realized variabliity while keeping the total amount of
content reasonable.

The resulting ink file transcription from the first episode traversal
can be found here: \url{http://lucidbard.com/ink/TWAU_001.ink}, while the
two SIG encodings can be found \url{http://lucidbard.com/ink/TWAU_001.vgl}
and \url{http://lucidbard.com/ink/TWAU_002.vgl}. David Elson's Scheherazade
v0.33 can be downloaded at
\url{http://www.cs.columbia.edu/~delson/software.shtml}.

\subsection{Encoding a Story Intention Graph}
\label{sec:orgheadline6}
When starting out, our first impulse was to encode every possible
detail of the dialogue exchanges. This often includes subtelty as in
any well written dialogue, using a third topic to communicate feelings
and status. One example of these sophisticated exchanges is in the
first epsiode when Bigby encounters Grendel outside of the business
office:

\begin{verbatim}
GRENDEL: What're you-- blind?
GRENDEL: What? You don't see there's a line? 
GRENDEL: I been standing here a half hour already. 
GRENDEL: You get to just walk in?
GRENDEL approaches BIGBY
GRENDEL: Must be nice being the Sheriff... 
GRENDEL: Do whatever the fuck you like.
- I don't have time for this.
- I'm cutting the line.
- [I work here.]
BIGBY: I work here.
GRENDEL: And what great work you do, Sheriff.
BIGBY: Hmm... that didn't feel very genuine.
BIGBY goes inside the office while GRENDEL looks on
 as the door shuts behind him.
GRENDEL: Fucker...
\end{verbatim}

This exchange is intended to communicate the privilege of the Sheriff,
as well as the distinction between the haves and have nots of the
community. The literal meaning is quite meaningless -- Grendel knows
that Bigby is Sheriff, as well as why he can enter the office. But he
still wants to air his feelings. It is also difficult to model what is
not said, such as in the next segment where Snow White and Bigby spend
a significant amount of time working on a murder case without a single
word exchanged about those waiting outside. In this case, also, the
goal of Grendel is unclear -- we don't know why Grendel is at the
business office, or why he is trying to antagonize Bigby. David Elson
designed the Story Intention Graph encourage a connection from every
action to a "core story goal" or value such as ego or health, but in
the nuanced and often ambiguos world of \emph{The Wolf Among Us}, such
intentions are not even clear after the final curtain.

\subsection{Text from Visual}
\label{sec:orgheadline7}
Many times the term text is applied to visual works in to describe the
accessibility and the coherence of the work. \emph{The Wolf Among Us} is
not the same work in a textual transcript than it is as a mimetic
video game. By performing a transcription of the content, including
the manner, props, and as much meaning as possible, it became
increasingly clear just how large a role performance played in the
conveyance of meaning in the work, and how utterly insufficient Story
Intention Graphs were to this nonverbal level of meaning.

When Colin gives Bigby a look after being asked to get out of the
chair, the underlying motivation is clear. But that motivation is
difficult to transcribe, or even to put into words. The familiarity is
an emotional state that colors the entirety of the interactions, but
the emotion that is present in those moments is not a simple state to
be started and stopped in the encoding of Story Intention Graphs.

This difficulty would equally well apply to encoding other non-textual
works. A given expression can be a lossy way to communicate, and often
times that message isn't required to understand a character's desires
or meaning. 

One potential solution that we had examined before was to retain all
of the non-textual information in the encoded format. This would
enable annotators full access to the expressions, the vocal
performance and the cinematography of the original. The burden of
assembling all of the video traversals, however, is very large and
still requires textual encoding for various natural language tools to
be able to act on or assist in the encoding process.

\subsection{Interpreting Content in \emph{The Wolf Among Us}}
\label{sec:orgheadline12}
Mawhorter goes into detail on different idioms of choices that covers
a broad swath of possible situations. Choice is used in varying
aesthetic and functional ways throughout \emph{The Wolf Among Us}. This
section describes the various ways that the choices function within
the work, informed by the different granularities and the insights
gleaned both by the rational reconstruction and encoding. 

Several decisions that define Telltale's style also have impact on the
choices above and beyond their literal meaning and status as choice as
described below. One of them is the fact that most decisions are
timed. This provides the player with a limited window of opportunity
to make a decision, and also limits the total decision space as a
result of requiring the player to read and understand all of the
available decisions. Rarely are there more than 4, and more often in
critical cases there are only two options. When reviewing a textual version of the work, th

Another decision made by the creators is to strongly signal to the
player the results of decisions after them through text. This text is
often referred to by the most famous occurence from Telltale's popular
series \emph{The Walking Dead}: "Clementine will remember that". Telltale
CEO Kevin Bruner, a keynote at ICIDS 2017, asserts that the text has
absolutely no correspondence with any simulation of mental models
within the game. Instead, they are purely designed to heighten the
illusion of agents considering and reacting to decisions that the
player has made. This addresses one of the shortcomings that Mateas
and Stern observed in the interactive drama \emph{Facade}\cite{Mateas2005},
where players were unable to track the state of the system through
purely naturalistic signals such as expressions and dialogue. The
reactions signaled are often more emotional than ontological, whether
in terms of representing a damaged relationship with Snow White over
brutalizing Mr. Toad or the appreciation for a kind gesture. These
responses become their own reward and heighten the already prevalent
feedback the game provides to social actions.

\subsubsection{Finding Character Between the Choices}
\label{sec:orgheadline8}
Several choice responses provide additional conversation turns where
the player-character performs actions after a player-designated choice
that are contrary to a player's intention, but which serve to ground
the character even in spite of player intentions. An example is in the
scene with Collin, when the player is presented with the following
choices (in INK format, where each choice is shown preceded by a '-'):

\begin{verbatim}
COLIN: I mean, look at your hands.
COLIN: Who'd you get in a fight with? 
COLIN: A Fable, right? I'm sure you're not going
       around punching Mundy's.

- [My job]
    BIGBY: I was doing by job, believe it or not.
    COLIN: Your job is to beat the shit out of fables?
    BIGBY: Sometimes I take them to The Farm.
    COLIN: Fuck you.
- [Not my fault]
- [Don't need advice]
- [...]
BIGBY: You think my job is easy? 
BIGBY: You try to keep a bunch of Fables from 
       killing each other. 
BIGBY: How do you think this all works?
\end{verbatim}

The response expected from Colin's followup to the initial response of
"My Job" is much harsher than one would expect, and further reflects a
deeper conflict in the protagonist toward his dual role of protector
and antagonist.

\subsubsection{Goals and Events}
\label{sec:orgheadline9}
The vast majority of speech acts, from a pragmatic perspective, do not
differ significantly from choice set to choice set. The narrative as
modeled is constructed mostly from the actions of others, and the
actions the protagonist, Bigby, takes of his own accord. Marie-Laure
Ryan describes two types of immersion that pertain to this type of
plot: "emotional immersion" for the sympathy with characters richly
described and acted, and "epistemological immersion" for the types of
questions that drive a mystery \cite{Ryan2009a}. The game provides a
set of clues along the way, but these clues are not gates to a next
piece of context. Identifying character goals is not part of the scene
where an investigation is conducted in Mr. Toad's apartment. Instead,
the entire focus of the player is on uncovering evidence of foul play,
and not questioning the motives behind it.

\subsubsection{Social Choices}
\label{sec:orgheadline10}
From each episode, between 6 and 8 decisions are tabulated and
presented to the player along with their own decision. One example of
this is whether the player gave Faith money at the beginning of the
episode. The fact that the player gave Faith money only impacts
whether the player is able to pay for a drink at the Trip Trap later
on in the epsiode, but most chose to give Faith money. One of the
stated reasons that the streamer provided to the anticipated
objections of his audience was that if the character asking for money
had been a poor man, then the results would not have been the
same. This motivation reveals that without knowing why a player made a
choice, it is easy to make false assumptions about their play style or
their character.

\subsubsection{[\ldots{}]}
\label{sec:orgheadline11}

One of the hard fast rules of Telltale games is the omnipresent silent
choice. It appears to serve three primary functions within \emph{The Wolf
Among Us}:

\begin{enumerate}
\item Often silence or a non-answer can elicit a more interesting
projection on the part of the conversation partner.
\item It allows players slop room for making a decision while keeping the
narrative going.
\item In the case of certain decisions, it allows the player an option to
opt out. One example is the decision of who to name as a primary
suspect.
\end{enumerate}

Ambiguity is always present in life. In modeling the dialogue choices
using Story Intention Graphs, the triple dot silence was the source of
the most agony. Not only was the ontological status of the text in
question, but the possible motivations varied with every single
player. Without a means of specifying multiple motivations that are
distinct, the system was unable to represent the null speech act. In
one case the silence resulted in a firm insult where the streamer had
not intended one. When a choice surprises a player, how is that
encoded? Does the player need to have a separate entity status, apart
from the protagonist? And does this even stop at the player -- does
the authorial goals deserve treatment as a mind reading on the part of
the player?

Story Intention Graphs provide a means of mapping out clear intentions
of agents and even interpretations of what happened, but one of the
fundamental tasks of game design is to anticipate multiple player
types and to understand how their motivations interact with the
possibilities in front of them. In the Wolf Among Us, violent options
are presented regardless of how nice the player has been before,
enabling people to change their mind, but also preventing genuine
character change.

\section{Future Work}
\label{sec:orgheadline14}
The primary goal of guiding narrative criticism with computational
models of narrative is to discover insights that might lie hidden
beneath the surface experience, but which underlies authorial
decisions. This paper's results motivates the work ahead in creating
the datasets to refine both the models of narrative as well as use
them in improving the experience of authoring complex, interactive
narratives.

Collecting a set of real-world traversals of a single interactive
fiction work would enable a variety of studies of emotion and
narrative to be done. Telltale has recently described some aspects of
its telemetry collection and the fact that the traversals serve as a
means of fingerprinting a particular character a player has
experienced. In Telltale's first season of \emph{Batman}, this is more
overtly described to the player at the end as a set of dimensions on
which your choices left a trail.

The next step is to collect additional natural traversals and
associated information about "why" decisions were made. These will
provide insight into just which aspects of the story were understood
and how different players experience even the same sequence of content
significantly differently based on their own personalities. Because of
the timing and the inability to see everything, we hypothesize that
the rationales will be as revealing as modeling the underlying
narrative motivations of the characters themselves.
\section{Conclusion}
\label{sec:orgheadline15}
This paper documents lessons learned from encoding portions of a
commercial choice-based narrative using existing computational models
designed for text-based narratives. We believe that the availability
of sources that can be annotated and studied will open the way for
improvements in assessment of narrative experiences as well as
improvements in authoring tools designed to support creation of
interactive digital narratives..
