\documentclass[sigconf]{acmart}

\usepackage{booktabs} % For formal tables
\usepackage{epstopdf}

% Copyright
%\setcopyright{none}
%\setcopyright{acmcopyright}
%\setcopyright{acmlicensed}
\setcopyright{rightsretained}
%\setcopyright{usgov}
%\setcopyright{usgovmixed}
%\setcopyright{cagov}
%\setcopyright{cagovmixed}


\begin{document}
\title{Applying story intention graphs to model narrative meaning of a non-linear cinematic adventure game}

\author{John Murray}
\email{lucid@soe.ucsc.edu}
\affiliation{
  \institution{University of California, Santa Cruz}
  \streetaddress{P.O. Box 1212}
  \city{Santa Cruz} 
  \state{California} 
  \postcode{95060}
}
\author{Michael Mateas}
\email{mmateas@soe.ucsc.edu}
\affiliation{
  \institution{University of California, Santa Cruz}
  \streetaddress{P.O. Box 1212}
  \city{Santa Cruz} 
  \state{California} 
  \postcode{95060}
}
\author{Noah Wardrip-Fruin}
\email{nwf@soe.ucsc.edu}
\affiliation{
  \institution{University of California, Santa Cruz}
  \streetaddress{P.O. Box 1212}
  \city{Santa Cruz} 
  \state{California} 
  \postcode{95060}
}

\begin{abstract}
Telltale games live in a zone untouchable by most modern tools of game
analysis. They reside firmly on the story side of the ludic-narrative
divide, and rely primarily on natural language with (relatively)
unsophisticated computational simulations of physics or other
systems. This paper describes an effort at applying an existing
modeling approach to a textual translation of the first episode of
Telltale Game's \emph{The Wolf Among Us} using a model of story content and
a freely available representation of interactive fiction. This
approach constituted one of the first such to apply narrative modeling
techniques to existing commercial games, and as such revealed a number
of issues with both the method of translation and the limitations of
the schemata selected in addressing interactive narratives. The
contribution can be summed up as the value of a medium "read" which is
distinct both from the idea of a distant read described by FILL IN and
from the common method of close play that has much in common with a
close reading used in literary analysis. This is accomplished through
taking a model of the narrative content to reveal the landscape of
core player experiences that make up the tapestry of individual
choice-driven gameplay traversals. David Elson developed the Story
Intention Graph schemata \cite{Elson2012} to capture discourse relationships of story
elements.
\end{abstract}

%
% The code below should be generated by the tool at
% http://dl.acm.org/ccs.cfm
% Please copy and paste the code instead of the example below. 
%
\begin{CCSXML}
<ccs2012>
 <concept>
  <concept_id>10010520.10010553.10010562</concept_id>
  <concept_desc>Computer systems organization~Embedded systems</concept_desc>
  <concept_significance>500</concept_significance>
 </concept>
 <concept>
  <concept_id>10010520.10010575.10010755</concept_id>
  <concept_desc>Computer systems organization~Redundancy</concept_desc>
  <concept_significance>300</concept_significance>
 </concept>
 <concept>
  <concept_id>10010520.10010553.10010554</concept_id>
  <concept_desc>Computer systems organization~Robotics</concept_desc>
  <concept_significance>100</concept_significance>
 </concept>
 <concept>
  <concept_id>10003033.10003083.10003095</concept_id>
  <concept_desc>Networks~Network reliability</concept_desc>
  <concept_significance>100</concept_significance>
 </concept>
</ccs2012>  
\end{CCSXML}

\ccsdesc[500]{Computer systems organization~Embedded systems}
\ccsdesc[300]{Computer systems organization~Redundancy}
\ccsdesc{Computer systems organization~Robotics}
\ccsdesc[100]{Networks~Network reliability}

% We no longer use \terms command
%\terms{Theory}

\keywords{ACM proceedings, \LaTeX, text tagging}


\maketitle

\input{fdg_paper}

%%% -*-BibTeX-*-
%%% Do NOT edit. File created by BibTeX with style
%%% ACM-Reference-Format-Journals [18-Jan-2012].

\begin{thebibliography}{00}

%%% ====================================================================
%%% NOTE TO THE USER: you can override these defaults by providing
%%% customized versions of any of these macros before the \bibliography
%%% command.  Each of them MUST provide its own final punctuation,
%%% except for \shownote{}, \showDOI{}, and \showURL{}.  The latter two
%%% do not use final punctuation, in order to avoid confusing it with
%%% the Web address.
%%%
%%% To suppress output of a particular field, define its macro to expand
%%% to an empty string, or better, \unskip, like this:
%%%
%%% \newcommand{\showDOI}[1]{\unskip}   % LaTeX syntax
%%%
%%% \def \showDOI #1{\unskip}           % plain TeX syntax
%%%
%%% ====================================================================

\ifx \showCODEN    \undefined \def \showCODEN     #1{\unskip}     \fi
\ifx \showDOI      \undefined \def \showDOI       #1{{\tt DOI:}\penalty0{#1}\ }
  \fi
\ifx \showISBNx    \undefined \def \showISBNx     #1{\unskip}     \fi
\ifx \showISBNxiii \undefined \def \showISBNxiii  #1{\unskip}     \fi
\ifx \showISSN     \undefined \def \showISSN      #1{\unskip}     \fi
\ifx \showLCCN     \undefined \def \showLCCN      #1{\unskip}     \fi
\ifx \shownote     \undefined \def \shownote      #1{#1}          \fi
\ifx \showarticletitle \undefined \def \showarticletitle #1{#1}   \fi
\ifx \showURL      \undefined \def \showURL       #1{#1}          \fi
% The following commands are used for tagged output and should be
% invisible to TeX
\providecommand\bibfield[2]{#2}
\providecommand\bibinfo[2]{#2}
\providecommand\natexlab[1]{#1}
\providecommand\showeprint[2][]{arXiv:#2}

\bibitem[\protect\citeauthoryear{Elson}{Elson}{2012}]%
        {Elson2012}
\bibfield{author}{\bibinfo{person}{David~K. Elson}.}
  \bibinfo{year}{2012}\natexlab{}.
\newblock {\em \bibinfo{title}{{Modeling Narrative Discourse}}}.
\newblock PhD Dissertation. \bibinfo{school}{Columbia University}.
\newblock
\showURL{%
\url{http://scholar.google.com/scholar?hl=en}}


\bibitem[\protect\citeauthoryear{Finlayson}{Finlayson}{2013}]%
        {Finlayson2013}
\bibfield{author}{\bibinfo{person}{Mark~Alan Finlayson}.}
  \bibinfo{year}{2013}\natexlab{}.
\newblock \showarticletitle{{A Survey of Corpora in Computational and Cognitive
  Narrative Science}}.
\newblock \bibinfo{journal}{{\em Sprache und Datenverarbeitung (International
  Journal for Language Data Processing)\/}} \bibinfo{volume}{37},
  \bibinfo{number}{1-2} (\bibinfo{year}{2013}), \bibinfo{pages}{113--141}.
\newblock
\showURL{%
\url{http://www.uvrr.de/index.php/anglistik/SDV}}


\end{thebibliography}
\end{document}
