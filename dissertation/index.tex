% Intended LaTeX compiler: pdflatex
\documentclass[11pt]{article}
\usepackage[utf8]{inputenc}
\usepackage[T1]{fontenc}
\usepackage{graphicx}
\usepackage{grffile}
\usepackage{longtable}
\usepackage{wrapfig}
\usepackage{rotating}
\usepackage[normalem]{ulem}
\usepackage{amsmath}
\usepackage{textcomp}
\usepackage{amssymb}
\usepackage{capt-of}
\usepackage{booktabs}
\usepackage{epstopdf}
\usepackage{mathptmx}
\usepackage{pifont}
\usepackage[margin=0.7in]{geometry}
\newcommand{\point}[1]{\noindent \textbf{#1}}
\usepackage{hyperref}
\usepackage{csquotes}
\usepackage[mla]{ellipsis}
\parindent = 0em
\setlength\parskip{.5\baselineskip}
\author{John Murray}
\date{\today}
\title{Analyzing Interactive Narratives using Computational Models and Player Response Data}
\hypersetup{
 pdfauthor={John Murray},
 pdftitle={Analyzing Interactive Narratives using Computational Models and Player Response Data},
 pdfkeywords={},
 pdfsubject={},
 pdfcreator={Emacs 27.0.50 (Org mode 9.1.6)},
 pdflang={English}}
\begin{document}

\maketitle
\section{Introduction}
\label{sec:org2fabd7a}
Why do we need stories?

\cite{Mateas2002-dw}

\cite{Elson2012-pi}


Stories encode complex experiential information in a way that not only
allows for communication, but also for living. They combine dynamic,
interesting situations with characters that are relatable and worth
getting to know. They capture the types of emotions that we don't
ordinarily get to experience in everyday life. The art of writing a
good story is to find the balance of finding true observations about
how humans live and connecting them together. We don't always have all
the information, we often are attached to outcomes, we take risks, we
make mistakes.

Existing models of story understanding pursue the "content" of stories
as if there were a simple model for how they exist in our
heads. Stories are not modeled, however, in the head of a reader. They
are lived, and then the results are more or less memorable. A
character's actions accumulate over time and over key sceenes to
become predictable. We come to know the characters in stories \emph{in the
very same way} as we have come to know people in our own lives.

What role do computers have to play in story? Computers can be
collaborators or research assistants; they can be actors or
directors. They can create stories using some of the same creative
algorithms that writers do.

What can computers not do?

They cannot predict, given a story, how a person will react. They
cannot tease apart the meaning from a passage. They cannot create a
moment that resonates in the way that a lifetime of witnessing human
events and people can.

And that seems pretty obvious, given that writers spend years watching
people and building various models of behavior and personality.

The act of witnessing a story is one of the most human activities in
existence, and one still far out of reach of current computers.

But our approaches to what is "computable" are beginning to shift. AI
has begun with a clean, mathematical model of meaning that translates
into planning, into ontologies and into a simpler model of how
information is accumulated and shared (schemas, scripts, etc). Today,
various neural network approaches are increasingly becoming capable of
tranining themselves on datasets in both supervised and unsupervised
methods. These systems are still far from being capable of reading
stories, and the research community has yet to realize exactly how
important it is to both have the capabliity of telling a story and to
make use of this most fundamental capacity to understand humans.

Stories are a lens into the human experience. This dissertation sets
out to rectify at least part of the two major trends of ai research
through the problem of stories. One, the statistical and data driven
approach of machine learning, and two, the logical representations of
meaning present in ontologies, diagrams and other innovations in
symbolic reasoning.

To do this, we'll attack a harder problem than story understanding as
it has yet been tackled, that of textual understanding. We'll develop
a theory of meaning co-creation in the tradition of reader response
using a case study of a choice-based cinematic adventure game that has
rich layers of authored and interpreted information. We'll also
likewise treat the player's experience as a primary source of insight
into the meaning of the work. Using the data including video recording
and phyisological signals from the experience itself and from the
player's raw moment-to-moment understanding and reaction we will begin
asking questions about what the right questions are.
\section{Story Understanding and The Semantic Gap}
\label{sec:org61c1160}

\section{Extending Story Intention Graphs}
\label{sec:orge99a8ec}

\section{Study Design \& Data Collection}
\label{sec:org4b54a99}

\section{Analytical Methodologies}
\label{sec:orgc2b755d}

\section{Conclusions}
\label{sec:orgd1e81c0}

\section{References}
\label{sec:org384c9d6}
\bibliographystyle{plain}
\bibliography{../../bibliography/references}

\section{Export Configuration}
\label{sec:org1cb39ee}
:noexport:ARCHIVE:
\end{document}
